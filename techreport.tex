\documentclass[english]{proposalnsf}
\usepackage{graphicx}
\usepackage{url}
\usepackage[square,numbers]{natbib}
\usepackage[nottoc,numbib]{tocbibind}

\title{An Ethnographic User Experience Evaluation of RadGrad}
\author{Quinne Uchida \\Collaborative Software Development Laboratory \\ Information and Computer Sciences \\ University of Hawaii}

\begin{document}
\maketitle
\tableofcontents
\newpage

\section{Introduction}
\label{introduction}

The demand for Computer Science jobs is growing rapidly and so is the demand for Computer Science degrees. However, employers are looking for individuals who have been educated in Computer Science, along with experience doing internships, extracurricular activities and other projects. The traditional perception of qualifications for obtaining a job were that one must have a high GPA. RadGrad is designed to address this problem. 

However, 65.7143{\%} of active students have low system adoption (see research design for more details) of RadGrad. 

In order to address the problem of low system adoption, this study aims to answer the question: "Why do/don't students use RadGrad?" by conducting a series of interviews with active students. 
Each student will be interviewed for 15 minutes, 3 times over the span of three weeks. Each interview's audo will be recorded and analyzed by descriptive and pattern coding (as described by The Coding Manual for Qualitative Researchers). 
After, all the interviews have been conducted and all of the audio analyzed, the researcher will present the most commonly used codes and the most commonly used patterns. 


\section{Related Work}
\label{related-work}

A qualitative user evaluation study has not yet been conducted on RadGrad. In the summer of 2019, a quantitative user evaluation study was conducted (do I reference the RadGrad SIGSE paper if it has not been published but it is going to be by the time I submit this propsal?). 

\section{Research Design}
\label{research-design}

We would like to improve the student participation in RadGrad, and by conducting an ethnographic study we may be able to identify and examine barriers of adoption for RadGrad. The main question I will be attempting to answer is: "Why do/don't students use RadGrad?"

In this study, the researcher will attempt to interview a total of 5-10 students. For each student, the researcher will interview the student for 15 minutes and record audio of the process. After the interview, the researcher will then transcribe and analyze the contents of the interview. The researcher will conduct 3 interviews per student, over the span of three weeks.  

We define low system adoption as 0-1 sessions per semester. We define a moderate level of adoption as 2-3 sessions per semester.  We define a high level of system adoption as 4 or more sessions per semester. 
According to data collected from RadGrad1 for Spring 2019 (2019-01-01 to 2019-05-31), out of 210 active students, 138 students are classified as having low system adoption. 57 students are classified as having moderate system adoption and 15 classify as having high adoption. In terms of percentage, 65.7143{\%} have low adoption, 27.1429{\%} have moderate adoption and 7.1429{\%} have high adoption. These numbers have been calculated from session data for Spring 2019 (2019-01-01 to 2019-05-31) obtained on 2019-11-20 as reported by the RadGrad system. 

{\em Sessions} are classified as periods of activity deliminated by 1 hour of inactivity. Having an hour of idle time creates a new session. The purpose of defining levels of adoption by number of sessions is because a user's number of login events may not accurately reflect activity of the system. 

While the researcher must allow the interview to change topics if the participant wishes, the ethnographer must still have a plan. Initially, the interviewer will ask the participant about their opinions on RadGrad. From there, the participant has the freedom to talk about any aspect of RadGrad and not a specific category. Theoretically, the first aspect of RadGrad the participant brings up will be the one deemed most important by that participant. 

Each student will be interviewed for 15 minutes, 3 times over the span of three weeks. Each interview's audio will be recorded. The audio recordings will be analyzed by descriptive and pattern coding (as described by The Coding Manual for Qualitative Researchers).

{\em Pilot Study}

Ethnographic studies are rarely done because of the time commitment required to analyze ethnographic data. I propose conducting a pilot study on 2 students. These students will be subject to the proposed structure of this study and will allow the ethnographer to further refine the proposed process. 

Incentives for participation are to be determined.

\section{Brief Summary}
{\em Summary}
{\em Purpose}
{\em Procedures}
{\em Background}


\section{Results}
\label{results}


\section{Conclusions}
\label{conclusions}



\bibliography{techreport}
\bibliographystyle{plainnat}

\appendix
\section{Formatting tips}


\end{document}


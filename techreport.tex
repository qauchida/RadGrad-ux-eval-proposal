\documentclass[english]{proposalnsf}
\usepackage{graphicx}
\usepackage{url}
\usepackage[square,numbers]{natbib}
\usepackage[nottoc,numbib]{tocbibind}
\title{A User Experience Evaluation of RadGrad}
\author{Quinne Uchida \\Collaborative Software Development Laboratory \\ Information and Computer Sciences \\ University of Hawaii}

\begin{document}
\maketitle
\tableofcontents
\newpage

\section{Introduction}
\label{introduction}

The demand for Computer Science jobs is growing rapidly and so is the demand for Computer Science degrees. However, employers are looking for individuals who have been educated in Computer Science, along with experience doing internships, extracurricular activities and other projects. The traditional perception of qualifications for obtaining a job were that one must have a high GPA. RadGrad is designed to address this problem. 

However, 65.7143{\%} of active students have low system adoption (see research design for more details) of RadGrad. 

In order to address the problem of low system adoption, this study aims to answer the question: "Why do/don't students use RadGrad?" by conducting a series of interviews with active students. 
The students will be split into 3 groups based on their adoption level of high, mid or low (for Fall 2019). The student will not be informed which adoption group they are in. Each student will be interviewed and asked the same questions individually for approximately 20 minutes during which the audio of the researcher and the student will be recorded. During the interview, the researcher will as a series of open ended questions to the student. After all the interviews have been conducted, the audio will be transcribed and then analyzed for key words and phrases.

It is likely that students with low to mid level adoption levels do not think that RadGrad is worth the time investing into using the system. For those people, they may view RadGrad as a worse version of STAR, the degree planner used by students at UH Manoa. College students have very little spare time or attention to give and are therefore apprehensive about investing time or attention into something they do not deem directly beneficial to them. 
Full time students are required to take 12-18 credits, which, given the recommended time ratio of credit hours and 2-3 hours spent studying, totals to 36-72 hours spent on academics alone. 

This study aims to come up with tangible reasons as to why students are not using RadGrad. RadGrad was launched in Spring 2018, and has been running for approximately 2 years. During the two year period between the launch and the present, the developers have introduced ways to capture usage statistics. By analyzing the captured statistics and conducting a qualitative study, this research aims to provide insight as to why students do not use RadGrad. These observations, in turn, could aid the software development process. 

\section{Related Work}
\label{related-work}

In the summer of 2019, a survey on student use of RadGrad was conducted. 
\section{Research Design}
\label{research-design}

We would like to improve the student participation in RadGrad, and by conducting this study we may be able to identify and examine barriers of adoption for RadGrad. The main research question I will be attempting to answer is: "Why do/don't students use RadGrad?"

We define low system adoption as 0-1 sessions per semester. We define a moderate level of adoption as 2-3 sessions per semester.  We define a high level of system adoption as 4 or more sessions per semester. 
{\em Sessions} are classified as periods of activity deliminated by 1 hour of inactivity. Having an hour of idle time creates a new session. The purpose of defining levels of adoption by number of sessions is because a user's number of login events may not accurately reflect activity of the system. 

According to data collected from RadGrad1 for Spring 2019 (2019-01-01 to 2019-05-31), out of 210 active students, 138 students are classified as having low system adoption. 57 students are classified as having moderate system adoption and 15 classify as having high adoption. In terms of percentage, 65.7143{\%} have low adoption, 27.1429{\%} have moderate adoption and 7.1429{\%} have high adoption. These numbers have been calculated from session data for Spring 2019 (2019-01-01 to 2019-05-31) obtained on 2019-11-20 as reported by the RadGrad system. 

The researcher will attempt to choose students with no personal association to the researcher, in order to avoid biasing the student’s answer. Students chosen are required to have a RadGrad account from the Fall of 2019. Ideally, the researcher will 3 groups of 4-5 students. Groups are divided by their level of adoption as defined earlier. 
 
The researcher will interview 15-21 students and each student will be interviewed individually in person for a total of 20 minutes. All interview audio will be recorded. 
 
The interview will be approximately 20 minutes and will consist of a series of open ended questions pertaining to RadGrad, with the focus of figuring out why or why doesn't the student use RadGrad. The exact questions are yet to be determined.  
\\ 
{\em Pilot Study}

A pilot study will be conducted with 3-6 students. Ideally, there will be at least one student representing each adoption group. 

Incentives for participation are to be determined.

Each interview's audio will be transcribed to text. The coding analysis process will begin once all of the interviews have been conducted and transcribed. Words will be coded using a hierarchical frame.  
\\

\section{Results}
\label{results}

No results as of this time.

\section{Conclusions}
\label{conclusions}

No conclusion as of this time.

\bibliography{techreport}
\bibliographystyle{plainnat}

\appendix
\section{Formatting tips}


\end{document}


\documentclass[english]{proposalnsf}
\usepackage{graphicx}
\usepackage{url}
\usepackage[square,numbers]{natbib}
\usepackage[nottoc,numbib]{tocbibind}
\title{Why Don't Students Use RadGrad? A Qualitative Study of Factors That Inhibit RadGrad Adoption}
\author{Quinne Uchida \\Collaborative Software Development Laboratory \\ Information and Computer Sciences \\ University of Hawaii}

\begin{document}
\maketitle
\newpage

\section{Introduction}
\label{introduction}

The demand for Computer Science jobs is growing rapidly and so is the demand for Computer Science degrees. However, employers are looking for individuals who have not only  been educated in Computer Science, but who also have experience doing internships, extracurricular activities and other projects. The traditional perception among students of qualifications for obtaining a job were that one must have a high GPA, but now much more is required to be competitive for high quality graduate programs and industry positions. RadGrad is designed to address this problem. 

Unfortunately, despite the intended benefits of RadGrad, student adoption is still relatively low after its first year of deployment. Approximately 65{\%} of students use RadGrad less than two times a semester. 

In order to address the problem of low system adoption, this study aims to answer the question: ``Why don't students use RadGrad?'' I will investigate this question by conducting a series of interviews with currently enrolled ICS students who use RadGrad once or less per semester. 

For example, it is possible that students with low to mid level adoption levels do not think that RadGrad is worth the time investing into using the system. 
Full time students are required to take 12-18 credits, which, given the recommended time ratio of credit hours and 2 hours spent studying, totals to 36-54 hours spent on academics alone. 

This study aims to come up with tangible reasons as to why students are not using RadGrad. RadGrad was launched in Spring 2018, and has been running for approximately 2 years. During the two year period between the launch and the present, the developers have introduced ways to capture usage statistics. By analyzing the captured statistics and conducting a qualitative study, this research aims to provide insight as to why students do not use RadGrad. These observations could lead to design changes or other actions that ultimately improve the adoption rate of RadGrad.  

\section{Related Work}
\label{related-work}

After the deployment of the RadGrad system, 2 methods of data collection were implemented to gain insight into the reasons behind certain students' behaviour toward RadGrad \cite{Johnson_RadGrad_2020}.  

\section{Research Design}
\label{research-design}

We would like to improve the student participation in RadGrad, and by conducting this study we may be able to identify and examine barriers to adoption of RadGrad. The main research question I will be attempting to answer is: ``Why don't students use RadGrad?''

We define low system adoption as 0-1 sessions per semester. We define a moderate level of adoption as 2-3 sessions per semester.  We define a high level of system adoption as 4 or more sessions per semester. 
{\em Sessions} are classified as periods of activity delimited by 1 hour of inactivity. Having an hour of idle time creates a new session. The purpose of defining levels of adoption by number of sessions is because a user's number of login events may not accurately reflect activity of the system. 

According to data collected from RadGrad for Spring 2019 (2019-01-01 to 2019-05-31), out of 210 active students, 138 students are classified as having low system adoption. 57 students are classified as having moderate system adoption and 15 classify as having high adoption. In terms of percentage, 65.7143{\%} have low adoption, 27.1429{\%} have moderate adoption and 7.1429{\%} have high adoption. These numbers have been calculated from session data for Spring 2019 (2019-01-01 to 2019-05-31) obtained on 2019-11-20 as reported by the RadGrad system. 

This study will investigate why students don't use RadGrad by conducting a series of interviews with currently enrolled ICS students who use RadGrad once or less per semester. Each student will be interviewed and asked the same questions individually for approximately 20 minutes during which the audio of the researcher and the student will be recorded. During the interview, the researcher will ask a series of open ended questions to the student. After all the interviews have been conducted, the audio will be transcribed and then analyzed for key words and phrases.

The researcher will interview 15-21 students and each student will be interviewed individually in person for a total of 20 minutes. All interview audio will be recorded. 
 
Each interview will be approximately 20 minutes and will consist of a series of open ended questions pertaining to RadGrad, with the focus of figuring out why students do not use RadGrad.

{\em Questions}

\begin{itemize}
	\item What is your class standing?
	\item When did you start your ICS undergraduate curriculum?
	\item Last semester, how many times did you use RadGrad? 
	\item How many credits did you take last semester?
	\item On average, last semester, how many hours a week did you work?	
	\item Please describe your current work situation and how it relates to ICS.
	\item What kind of extracurricular activities do you participate in?
	\item Please describe you level of involvement of those extracurricular activities.
	\item When you think of career planning, what resources do you use?
	\item Where would you like to be in 5 years regarding your career?
	\item In the previous question, you stated where you would like to be in 5 years regarding your career. How do you plan to get to that place?
	\item What kind of internship opportunities are you interested in?
	\item Why did you choose to major in Computer Science?
	\item Please describe how useful you find RadGrad.
	\item How important do you believe GPA is to prospective recruiters?
	\item How will this experience help you in your future career?
	\item Have you talked with any mentors about your opportunities in ICS? Were your mentors professors, ICS alumni, academic advisors, friends?
	\item Was there a piece of advice you'd hear a lot of people give? If so, what is that advice?
	\item Do you feel that your degree plan has no direction? If so why?
	\item How often do you question why you are in college?
	\item What do you think the purpose of RadGrad is?

\end{itemize} 

{\em Sequence of Events}

\begin{enumerate}
	\item Obtain the anonymized user statistics of RadGrad for 2019-08-01 to 2019-12-31.
	\item Calculate the percentage of low-adoption users. If the percentage of low adoption users decreases to below 30{\%}, I may have to redesign my entire study. 
	\item Ask for email addresses of all low-adoption users.
	\item Send mass email for participation in a RadGrad User experience evaluation. Highlight that they have been contacted because RadGrad wants to understand \textbf{why people are not using it.}
	\item Pray that people respond.
	\item Conduct interviews.
	\item Transcribe results.
	\item Compose the schema of how the interview responses are interpreted.
	\item Analyze the results of the interviews and draw conclusions.
	\item Analyze the process of steps 1-9. If there are any enhancements the research design could benefit from, take those ideas into consideration for next iteration of the research.
\end{enumerate}

{\em Pilot Study}

A pilot study will be conducted with 3-6 students. Ideally, there will be at least one student representing each adoption group. The pilot study will follow the sequence of events listed above. After the pilot study, the research design and sequence of events may need to be altered for optimization purposes.  

\bibliography{techreport}
\bibliographystyle{plainnat}

\appendix
\section{Formatting tips}


\end{document}

